%%%%%%%%%%%%%%%%%%%%%%%%%%%%%%%%%%%%%%%%%%%%%%%%%%%%%%%%%%%%%%%
%
% Welcome to Overleaf --- just edit your LaTeX on the left,
% and we'll compile it for you on the right. If you open the
% 'Share' menu, you can invite other users to edit at the same
% time. See www.overleaf.com/learn for more info. Enjoy!
%
%%%%%%%%%%%%%%%%%%%%%%%%%%%%%%%%%%%%%%%%%%%%%%%%%%%%%%%%%%%%%%%
\documentclass{beamer}

\usetheme{Madrid}
\usecolortheme{lily}
\addtobeamertemplate{footnote}{}{\vspace{2ex}}

\usepackage{amsmath}
\usepackage{amssymb}
\usepackage{color}
\usepackage{listings}
\usepackage{clrscode3e}
\usepackage{multicol}

\definecolor{codegreen}{HTML}{237e02}
\definecolor{codegray}{rgb}{0.5,0.5,0.5}
\definecolor{codepurple}{HTML}{8F4673}
\definecolor{codebrown}{HTML}{ce9178}
\definecolor{codecyan}{HTML}{098658}
\lstdefinestyle{pythonstyle}{
    commentstyle=\color{codegreen},
    keywordstyle=\color{codepurple},
    numberstyle=\tiny\color{codegray},
    stringstyle=\color{codebrown},
    basicstyle=\ttfamily\small,
    breakatwhitespace=false,         
    breaklines=true,                 
    captionpos=b,                    
    keepspaces=true,                 
    numbersep=5pt,                  
    showspaces=false,                
    showstringspaces=false,
    showtabs=false,
    tabsize=2
}
\def\And{\text{ AND }}
\def\Or{\text{ OR }}
\def\Xor{\text{ XOR }}
\def\Implies{\text{ IMPLIES }}
\def\Iff{\text{ IFF }}
\def\Not{\text{NOT}}
\def\R{\mathbb{R}}
\def\N{\mathbb{N}}
\lstset{style=pythonstyle}

\setlength{\parskip}{1em}

%Information to be included in the title page:
\title{Format for Efficient Storage of Homology Relations}
\subtitle{Week 3 Report: Understanding Queries and Inference on Gene Trees}
\author{Kevin Gao}
\institute{University of Toronto}

\begin{document}

\frame{\titlepage}

\AtBeginSection[]
{
    \begin{frame}
    \frametitle{Outlines}
    \tableofcontents[currentsection]
    \end{frame}
}

\section{Orthologs and Paralogs}

\begin{frame}{Orthologs and Paralogs}
    Orthologs: Homologs separated by \textbf{speciation} events

    \begin{itemize}
        \item 1-to-1 orthologs: one to one pair
        \item 1-to-many orthologs: one gene is orthologous to many
        \item many-to-many
    \end{itemize}

    Paralogs: Homologs separated by \textbf{duplication} events

    \begin{itemize}
        \item same-species paralog: paralogs within the same species
        \item between-species paralog: paralogs in different species due to duplication in common ancestor
        \item fragments of the same gene: stored with separate labels on leaves?
    \end{itemize}
\end{frame}

\begin{frame}{Naive Algorithm for Inference}
    Boils down to ancestry query. Given two leaves, if they share a \textbf{common ancestor} in the gene tree, then they are homologous.

    If the \textbf{lowest common ancestor} is a duplication node, then the two genes are paralogs of each other. If the lowest common ancestor is a speciation node, then the two genes are orthologs of each other.

    Within/Between can be determined via a simple query of the species to which the genes belong. One-to-many/Many-to-many can be determined by counting the number of duplication nodes on the path from the LCA to the two leaves.
\end{frame}

\begin{frame}{A Speed-up}
    The worst case for the naive algorithm occurs when two nodes do not have a common ancestor (are not homologous), in which case we won't find that out until we reach the root.

    A string-based binary index like the one discussed last week can allow us to determine whether two nodes share a common ancestor without actually traversing the tree.
\end{frame}

\begin{frame}{Batch Queries}
    We should also consider batch queries and queries asking for ``all orthologs/paralogs/etc.'' of a given gene.

    For the ``list all'' type of queries, we look at event nodes instead of individual leaves. For a given gene represented as a leaf, to list all paralogs, we find all duplication events on the root-to-leaf paths. After we find the duplication event nodes, we list all leaves of the subtree rooted at these duplication nodes.
\end{frame}

\section{Impact on Choice for Formats}

\begin{frame}{If we use Newick}
    \begin{itemize}
        \item We would need some special label to indicate speciation and duplication node
        \item Convert an LCA query to an equivalent RMQ query
        \item Compact but less information
        \item Would still have to parse the text file into a tree in memory for more complex queries
    \end{itemize}
\end{frame}

\begin{frame}{If we use PhyloXML}
    \begin{itemize}
        \item VTD would be most ideal since it indexes parent-child-sibling relationship
        \item Implement along with a string-based index for ancestry relationship
        \item Streaming models can be used for extremely large data
        \item Well-established specification for including speciation and duplication events (e.g. recPhyloXML)
    \end{itemize}
\end{frame}

\section{References}

\begin{frame}
    \scriptsize

    T. C. Lam, J. J. Ding and J. Liu, "XML Document Parsing: Operational and Performance Characteristics," in Computer, vol. 41, no. 9, pp. 30-37, Sept. 2008, doi: 10.1109/MC.2008.403.

    Haim Kaplan, Tova Milo, and Ronen Shabo. 2002. A comparison of labeling schemes for ancestor queries. In Proceedings of the thirteenth annual ACM-SIAM symposium on Discrete algorithms (SODA '02). Society for Industrial and Applied Mathematics, USA, 954-963.

    L. Nakhleh, D. Miranker and F. Barbancon, "Requirements of phylogenetic databases," Third IEEE Symposium on Bioinformatics and Bioengineering, 2003. Proceedings., 2003, pp. 141-148, doi: 10.1109/BIBE.2003.1188940.

    Cardona, G., Rossello, F. and Valiente, G. Extended Newick: it is time for a standard representation of phylogenetic networks. BMC Bioinformatics 9, 532 (2008). https://doi.org/10.1186/1471-2105-9-532

    Kmettlca, E.A. O(log n) persistent online lowest common ancestor search without preprocessing. https://github.com/ekmett/lca/
    
    Wansong Zhang, Daxin Liu and Jian Li, "An encoding scheme for indexing XML data," IEEE International Conference on e-Technology, e-Commerce and e-Service, 2004. EEE '04. 2004, 2004, pp. 525-528, doi: 10.1109/EEE.2004.1287357.
\end{frame}

\end{document}